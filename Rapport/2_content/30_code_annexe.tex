\section*{Code annexe}
    \addcontentsline{toc}{section}{Code annexe}

\begin{lstlisting}[caption=Code : Variation des paramètres de la loi Poisson, label=lst:poisson]
n <- 100
lamda_1 <- 2
lamda_2 <- 5
poiss <- rpois(n, lambda = lamda_1)
poiss2 <- rpois(n, lambda = lamda_2)
df <- data.frame(value = c(poiss, poiss2), distribution = c(rep("lambda = 2", n), rep("lambda = 5", n)))
ggplot(df, aes(x=value, fill=distribution)) +  # Histogramme
  geom_histogram(aes(y=..density..), position="identity", alpha=0.5, bins=30) + 
  geom_density(alpha=0.5)

# Simulez differentes lois de Poisson en faisant varier lambda ainsi que la taille de l'echantillon
lambda <- 30 # paramètre de la loi de Poisson
n <- 1000
ech <- rpois(n,lambda) # simulation de la loi de Poisson
summary(ech)
barplot(table(ech))
\end{lstlisting}

\begin{lstlisting}[caption=Code : Vérification de l'égalité, label=lst:comparaison]
equation_1 <- integrate(function(x) exp(-x^2/2),-Inf,Inf)
equation_2 <- sqrt(2*pi)
ggplot(df, aes(x = x)) + # Tracer la fonction densité
  geom_line(aes(y = y, color = "f1"), alpha = 0.5, size = 1) +  # Courbe rouge avec transparence
  geom_line(aes(y = y2, color = "f2"), alpha = 0.5, size = 1) +  # Courbe bleue avec transparence
  labs(x = "x", y = "y", color = "Légende")
\end{lstlisting}

\begin{lstlisting}[caption=Code : Probabilités des lois normales, label=lst:theoreme]
n <- 2000
mu <- 1
sigma <- 4
ech <- rnorm(n, mu, sigma)
ggplot(df, aes(x = (x - mu) / sigma)) +
  geom_density(aes(color = "Densité simulée"), fill = "#00BFC4", alpha = 0.5) +
  stat_function(fun = dnorm, aes(color = "Densité théorique"), size = 1.2) +
  labs(x = "Valeurs", y = "Densité", color = "Légende")
\end{lstlisting}

\begin{lstlisting}[caption=Code : Probabilités des lois normales, label=lst:q1]
# A suivre du code TP, section : Exercices sur la loi normale, Question 1
print(colSums(nd_en_col< -2)/n * 2000)
print(colSums(nd_en_col< 0)/n * 2000)
print(colSums(nd_en_col==0)/n * 2000)
print(colSums(nd_en_col> 2)/n * 2000)
\end{lstlisting}

\begin{lstlisting}[caption=Code : Probabilités par fonctions de répartition, label=lst:q6]
pnorm(-1,0,2) # F(-1) : P(X <= -1)
pnorm(2,0,2) # F(2) : P(X <= 2)
1-pnorm(2,0,2) # 1 - F(2) : P(X > 2)
pnorm(2,0,2)-pnorm(-1,0,2) # F(2) - F(-1) : P(-1 < X <= 2)
\end{lstlisting}

\begin{lstlisting}[caption=Code : Lien entre les lois normales, label=lst:lien_normales]
y <- rnorm(10000, 4, sqrt(3)) # On simule des variables aléatoires gaussiennes
z <- (y - 2) / sqrt(3) # On applique la transformation linéaire
# Graphique de comparaison entre y et z
df <- data.frame(value = c(y, z), distribution = c(rep("Y", length(y)), rep("Z", length(z))))
ggplot(df, aes(x = value, fill = distribution)) +
    geom_histogram(aes(y = ..density..), position = "identity", alpha = 0.5, bins = 30) +
    geom_density(alpha = 0.5)
\end{lstlisting}

\begin{lstlisting}[caption=Code : Sommes de variables aléatoires gaussiennes, label=lst:somme_gaussiennes]
# Meme ecart type mais pas de meme esperance
lines(x,dnorm(x,1,sqrt(1*1 + 1*1)),col="blue4") #la loi proposée

# Meme esperance mais pas de meme ecart type
lines(x,dnorm(x,0,sqrt(1*1 + 2*2)),col="blue4") # la loi proposée
\end{lstlisting}

\begin{lstlisting}[caption=Code : Démonstration de la loi d'une moyenne de lois gaussienne $\bar{X}$, label=lst:mean_gaussiennes]
n <- 10000 # Nombre de simulations
x1<-rnorm(n,0,1) # On simule des variables aléatoires gaussiennes
x2<-rnorm(n,0,2)
x3<-rnorm(n,3,1)
x4<-rnorm(n,2,1)

x_bar <- (x1 + x2 + x3 + x4) / 4 # On calcule la moyenne de ces variables aléatoires

hypothese <- rnorm(10000,(0+0+3+2)/4, sqrt(1*1 + 2*2 + 1*1 + 1*1)/4) # On simule l'hypothse de la loi d'une moyenne de lois gaussienne (voir rapport section 2.8)

# Comparaison entre la moyenne et une loi normale
df <- data.frame(value = c(moyenne, hypothese), 
                    distribution = c(rep("Moyenne", length(moyenne)), rep("Hypothese", length(hypothese))))

ggplot(df, aes(x = value, fill = distribution)) +
    geom_histogram(aes(y = ..density..), position = "identity", alpha = 0.5) +
    geom_density(alpha = 0.5)
# Conclusion: On percoit bien que les deux densités se supperposent parfaitement.
\end{lstlisting}



\begin{lstlisting}[caption=Code : Exercice sur la loi normale, label=lst:exo]
moyenneH <- 172 # moyenne de la taille des hommes
varianceH <- 196 # variance de la taille des hommes
question1 <- pnorm(160, moyenneH, sqrt(varianceH)) # F(160)
question2 <- 1 - pnorm(200, moyenneH, sqrt(varianceH)) # 1 - F(200)
question3 <- pnorm(185, moyenneH, sqrt(varianceH)) - pnorm(165, moyenneH, sqrt(varianceH)) # F(185) - F(165)
moyenneF <- 162 # moyenne de la taille des femmes
varianceF <- 144 # variance de la taille des femmes
question4 <- 1 - pnorm(0, moyenneH - moyenneF, sqrt(varianceH + varianceF)) # 1 - F(0)
\end{lstlisting}



\begin{lstlisting}[caption=Code : Vérification de la densité de probabilité de la loi exponentielle, label=lst:exp]
x <- seq(0, 10, 0.1)
lambda <- 0.5
f <- lambda * exp(-lambda * x)
# Tracer la fonction densité avec ggplot
df <- data.frame(x = x, f = f)
ggplot(df, aes(x = x, y = f)) + geom_line(color = "blue")
# vérifier l'ntégration à 1
integrate(function(x) lambda * exp(-lambda * x), 0, Inf)  
\end{lstlisting}


