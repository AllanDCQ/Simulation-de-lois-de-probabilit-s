\section*{Code annexe}
    \addcontentsline{toc}{section}{Code annexe}

\begin{lstlisting}[caption=Code exemple : La loi normale, label=lst:normal_distribution]  
 #------------------------ Code exemple : La loi normale ------------------------
 # Args :
 #   n: taille de l'échantillon -> int
 #   ech: simulation de l'échantillon -> vecteur
 #   quantile.normale: quantile de la loi normale -> vecteur
 #   quantile.echantillon: quantile de l'échantillon -> vecteur
 #-------------------------------------------------------------------------------

 n<-30   #Taille de l'échantillon 30
 # Simulation d'une loi normale d'espérance 1 et de variance 4 (écart-type = 2)
 ech<-rnorm(n,mean = 0,sd = 2)

 # Graphique 1
     # Création de l'histogramme de ech
     hist(ech, freq=F)
     # Ajout de la courbe de densité de a loi normale
     curve(dnorm(x,mean=0,sd=2),from=-6,to=6,ylab="densité",add=T,col="red")
 # Graphique 2
     # Tracer le graphique de ech
     plot(ecdf(ech))
     # Tracer la fonction de répartition de ech
     curve(pnorm(x,mean=0,sd=2),from=-6,to=6,ylab="Fonction de répartition",add=T,col="red")

 # Création du quantile de la loi normale d'espérance 1 et d'écart-type = 2
 quantile.normale<-qnorm(p=c(0.05,0.1,0.25,0.5,0.75,0.9,0.95),mean = 0,sd = 2)
 # Création du quantile de la simulation "ech"
 quantile.echantillon<-quantile(x=ech, probs=c(0.05,0.1,0.25,0.5,0.75,0.9,0.95))
 # Tableau de comparaison des 2 quantiles
 (rbind(quantile.echantillon,quantile.normale))
\end{lstlisting}


\begin{lstlisting}[caption=Code exemple : Simulation de la loi uniforme discrète, label=lst:uniform_distribution]  
 #------------------- Code exemple : La loi uniforme discrete -------------------
 # Args :
 #   k: taille de l'échantillon -> int
 #   ech: simulation de l'échantillon -> vecteur
 #-------------------------------------------------------------------------------

  k <- 30  # Création de la taille de l'échantillon
  # Simulation d'un échantillon de taille n et de valeurs {1,...,10}.
  ech<-sample(1:10,k,replace=T)
  
  # Graphique 1
    # Création d'une graphique bar 
    barplot(table(ech)/k)
    # Trace de la probabilite de la loi uniforme discrete sur {1,...,10}.
    abline(h=1/10,col="red")
\end{lstlisting}



\begin{lstlisting}[caption=Code exemple : Simulation de la loi de Bernoulli, label=lst:bernoulli_distribution]  
 #---------------------- Code exemple : La loi de Bernoulli ---------------------
 # Args :
 #   n: taille de l'échantillon -> int
 #   p: paramètre de la loi de Bernoulli -> float
 #   esp: espérance théorique -> float
 #   var: variance théorique -> float
 #   ech: simulation de l'échantillon -> vecteur
 #-------------------------------------------------------------------------------

 n<- 30 # Taille de l'échantillon
 p<-0.3  # paramètre de la loi de Bernoulli
 esp<-p # Espérence théorique
 var<-p*(1-p) # espérance théorique

 # Simulation d'un échantillon de taille n selon une loi de Bernoulli
 ech<-rbinom(n,size = 1,prob = p) # size = 1 correspond à la loi de Bernoulli
 table(ech)/n # Affichage des fréquences observées
\end{lstlisting}


\begin{lstlisting}[caption=Code exemple : Simulation de la loi binomiale, label=lst:binomiale_distribution]  
 #---------------------- Code exemple : La loi de Bernoulli ---------------------
 # Args :
 #   N: nombre d'observations -> int
 #   n: nombre d'essais -> int
 #   p_1: probabilité succès du premier test -> float
 #   p_2: probabilité de succès du deuxième test -> float
 #   ech_1: simulation de l'échantillon pour le premier test -> vecteur
 #   ech_2: simulation de l'échantillon pour le deuxième test -> vecteu
 #   ech_3: simulation de l'échantillon pour le troisième -> vecteur
 #   ech_4: simulation de l'échantillon pour le quatrième -> vecteur
 #-------------------------------------------------------------------------------
   
 N <- 100 # taille de l'échantillon
 n <- 10  # premier paramètre de la loi

 p_1<- 0.8 # deuxième paramètre de la loi : probabilité de succès
 p_2<- 0.2 # deuxième paramètre de la loi : probabilité de succès

 ech_1<-rbinom(N,n,p_1) # Simulation d'un échantillon pour p=0.8
 ech_2<-rbinom(N,n,p_2) # Simulation d'un échantillon pour p=0.2

 # Affichage des la table de propotion de la simulation
 table(ech_1)/N ; table(ech_2)/N

 dbinom(x=0:n,size = n,prob = p_1) # Simulation de la densité de la loi binomiale pour p=0.8 
 dbinom(x=0:n,size = n,prob = p_2) #  Simulation de la densité de la loi binomiale pour p=0.2

 ech_3<-rbinom(10000,n,p_1) # Simulation d'un échantillon pour p=0.8 et N=1000
 ech_4<-rbinom(10000,n,p_2) # Simulation d'un échantillon pour p=0.2 et N=1000
  \end{lstlisting}

\begin{lstlisting}[caption=Code exemple : Simulation de la loi géométrique, label=lst:geometric_distribution]  
#----------------------- Code exemple : La loi Géométrique ----------------------
# Args :
#   n: nombre d'essais -> int
#   p: probabilité succès du premier test -> float
#-------------------------------------------------------------------------------
    
n <- 25 # taille de l'échantillon
p <- 0.3 # paramètre de la loi 
ech<-rgeom(n,p)+1 # simulation de la loi
barplot(table(ech))
\end{lstlisting}
